% compile this file with xelatex
\documentclass[12pt]{article}
\usepackage{graphicx}
\usepackage[margin=2cm, a4paper]{geometry}
\usepackage{setspace}
\usepackage{pdfpages}
\usepackage{float}
\usepackage{ctex}
\usepackage{amsmath}
\usepackage{fancyvrb}
\usepackage{amssymb}
\usepackage{minted}
% \usepackage{enumitem}
\usepackage[shortlabels]{enumitem}
\usepackage[colorlinks,linkcolor=blue]{hyperref}

\usepackage{xeCJK}
\setCJKmainfont{Noto Sans TC}

\renewcommand{\contentsname}{Contents}
\renewcommand{\figurename}{Figure}
\renewcommand{\tablename}{Table}
\hypersetup{
    colorlinks=true,
    linkcolor=black,
    filecolor=magenta,      
    urlcolor=blue,
}
\newcommand{\mytitle}{Week2 出題練習HW - 偉大性格的射手}
\newcommand{\myauthor}{第 16 組}
\usepackage{fancyhdr}
\pagestyle{fancy}
\fancyhead{}
\fancyhead[L]{\mytitle}
\fancyhead[R]{\myauthor}

\title{\mytitle}
\author{\textbf{\myauthor}}
\date{Due: 2025/9/17}

\begin{document}

\onehalfspacing
\maketitle

\section{Problem}
\subsection{說明}
今年是英雄聯盟中國賽區(LPL)最有希望的一年,身為熱門實況主的JackeyLove正在他人生中第一次的世界賽舞台上對決來自北美賽區(LCS)的 Team Liquid,在隊上擔任射手(AD carry)這個職位的他,對於金錢的運用一定要嚴謹,與對手對線到殘血的他正要回城出裝,因為身上的錢 $M$ 不足以出一件大裝,但為了提升自己的能力而需要出兩件小裝作為過渡,商城裡總共有 $N$ 件小裝,價格分別為 $n_0,n_1,n_2,\dots,n_{N-1}$,為了讓自己數值最大化,請找出兩件裝備的價格之和等於自己身上的錢 $M$ 的唯一組合。
\subsection{Input Format and Constraints}
\begin{itemize}
    \item $2 \le N \le 5*10^4$
    \item $0\le n_0,n_1,\dots,n_{N-1} < M\le10^9$
    \item 第一行有兩個數字 $N$ 與 $M$ 用一個空格隔開,分別代表商城內的裝備總數與自己身上有多少錢。
    \item 第二行有 $N$ 個數字 $n_0,n_1,n_2,\dots,n_{N-1}$ 代表這 $N$ 件裝備的售價,且每件裝備的售價都不一樣。
\end{itemize}
\subsection{Output Format}
只有一行,由小到大輸出兩件裝備的 index $A$ 與 $B$ (0-based),使得 $n_A+n_B=M$  為唯一解,$A$ 與 $B$ 之間用一個空格隔開。
\subsection{Sample Input 1}
\begin{verbatim}    
3 5
2 3 1
\end{verbatim}
\subsection{Sample Output 1}
\begin{verbatim}
0 1
\end{verbatim}
\subsection{Sample Input 2}
\begin{verbatim}
5 10
5 9 6 3 4
\end{verbatim}
\subsection{Sample Output 2}
\begin{verbatim}    
2 4
\end{verbatim}
\subsection{Sample Input 3}
\begin{verbatim}
10 20
19 10 2 4 17 3 12 15 6 9
\end{verbatim}
\subsection{Sample Output 3}
\begin{verbatim}
4 5
\end{verbatim}
\section{Solution}
這題的核心觀念是使用一個類似 hash table 功能的陣列 $ht$,來紀錄每個元素的 index (因為根據題目,每個元素都會是 unique 的)。

但此題的 $n_i$ 太大了,所以我們需要透過一些手段來節省使用的 memory,在官方解答中,我們會找出 $n_i$ 中的 minimum 和 maximum,然後存每個元素的 index 時,扣掉 minimum (後面稱作 offset)。例如我們存 $n_i$ 的 index 為 $i$ 時,會紀錄為 $ht[n_i-offset]=i+1$。這樣可以讓 \texttt{calloc} 的大小限制在 \texttt{maximum-minimum+1} (單位是 \texttt{int})

最後,我們 iterate 過陣列的每個元素,然後看 index 是 $m-n_i-offset$ 的元素是否存在 (即不為 $0$,因為 \texttt{calloc} 會給 $ht$ 每個元素0的初始值,所以是0代表沒這個元素) ,且 index 不為 $i+1$ (即 $n_i$ 自己),如果存在即回傳該答案。

整體而言,此演算法的時間複雜度會是 $O(N)$,因為我們必須遍歷過常數次的陣列,空間複雜度則為 $O(\max{n_i}-\min{n_i})$,$\max{n_i}$ 與 $\min{n_i}$ 分別為 $n_i,i\in\mathbb{N},i\in[0,N-1]$ 的最大值與最小值,因為我們必須建立 hash table $ht$,其大小為 $\max{n_i}-\min{n_i}+1$ 單位的 32-bit 整數。
\section{Code}
\begin{minted}[frame=lines,framesep=2mm,baselinestretch=1.2,linenos,breaklines]{C}
#include <stdio.h>
#include <stdlib.h>

int *twoSum(int *nums, int n, int m)
{
    int max = nums[0], min = nums[0];
    for (int i = 1; i < n; i++)
    {
        if (nums[i] > max)
            max = nums[i];
        else if (nums[i] < min)
            min = nums[i];
    }
    int htsize = max - min + 1;
    int *ht = calloc(htsize, sizeof(int));
    int offset = min;

    for (int i = 0; i < n; i++)
    {
        ht[nums[i] - offset] = i + 1;
    }
    int *retarray = malloc(2 * sizeof(int));
    for (int i = 0; i < n; i++)
    {

        int leftidx = m - nums[i] - offset;
        if (leftidx >= 0 && leftidx < htsize && ht[leftidx] && ht[leftidx] != (i + 1))
        {
            retarray[1] = ht[leftidx] - 1;
            retarray[0] = i;
            free(ht);
            return retarray;
        }
    }
    return 0;
}
int main()
{
    int n, m;
    scanf("%d%d", &n, &m);
    int *nums = malloc(n * sizeof(int));
    for (int i = 0; i < n; i++)
    {
        scanf("%d", &nums[i]);
    }
    int *ans = twoSum(nums, n, m);
    printf("%d %d\n", ans[0], ans[1]);

    return 0;
}
\end{minted}
\section{TestData}
本題我們共安排 7 筆測試資料,它們被設計的目的如下所示:
\begin{enumerate}[(a)]
    \item subtask 0-3 (\texttt{[0-3].in}): 檢驗程式基本邏輯是否正確、能否給出正確的輸出。(Brute force is acceptable.) (50\%,$N<100,M<5*10^4,n_i<10^4$)
    \item subtask 4-6 (\texttt{[4-6].in}): 輸出正確以外,也檢驗程式的空間安排是否正確,在 $N$ 與 $n_i$ 較大的情況下,不只使用 hash table 的概念來改善時間複雜度,也有使用 solution 中 offset 的手法來節省空間以符合 memory 的規範。(50\%, 除測資符合題目範圍外無任何限制)
\end{enumerate}
\section{Teamwork}
\begin{itemize}
    \item 題目設計/包裝與題解: 林有諒、陳昫叡
    \item 測資設計、驗題: 李國鈺、吳宥勳
    \item 校對測試、書面報告: 賴昱錡
\end{itemize}
\section{Reference}
本題的發想來自 \href{https://leetcode.com/problems/two-sum/description/}{LeetCode 1.Two Sum},但修改了原本題目的條件 (每個元素變成都要是 unique 的)、constraints、測資範圍與題目包裝。



\end{document}
% how to display codes?
% \begin{minted}[frame=lines,framesep=2mm,baselinestretch=1.2,linenos,breaklines]{python}
% \end{minted}

% how to display images?
% \begin{figure}[H]
%     \centering
%     \includegraphics[width=0.5\linewidth]{}
%     \caption{Caption}
% \end{figure}
% test